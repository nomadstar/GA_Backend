\section{Introducción}

\subsection{Resumen}
Este trabajo recoge y continúa avances previos en la construcción de herramientas para la planificación académica. La implementación original conocida como \texttt{RutaCritica} (implementada en Python) aportó pilares conceptuales y prácticos decisivos: el uso de un análisis PERT para identificar ramos críticos, la codificación de una función de prioridad posicional (concatenación CC+UU+KK+SS) y una formulación basada en grafos donde las secciones son nodos y las compatibilidades definen aristas. \texttt{RutaCritica} resolvía el problema mediante construcción explícita de grafos y búsqueda de cliques ponderados con librerías como NetworkX, lo que proporcionó una línea base exhaustiva y altamente expresiva del dominio.

Sobre esa base conceptual se desarrolló \texttt{Quickshift}, una reimplementación y evolución diseñada para producción en tiempo real. \texttt{Quickshift} traduce las ideas principales de \texttt{RutaCritica} al ecosistema Rust y modifica la estrategia algorítmica: en vez de confiar en búsquedas exactas y exhaustivas (que originaban cliques de cardinalidad mayor y posterior poda destructiva), adopta una heurística constructiva voraz con diversificación por semillas múltiples (greedy multi-seed). Esta aproximación conserva los criterios de prioridad y las comprobaciones PERT/prerrequisitos heredadas de \texttt{RutaCritica}, pero optimiza la construcción de soluciones para reducir uso de memoria y latencia, e incorpora mecanismos adicionales de normalización y mapeo de equivalencias entre códigos de asignatura.

Las decisiones de diseño son explícitas: respetar y reconocer la aportación técnica de \texttt{RutaCritica} (PERT, prioridad posicional, modelado por grafos) y, al mismo tiempo, introducir optimizaciones algorítmicas y de ingeniería (Rust, gestión determinista de secciones, normalización, pruebas automatizadas) que hacen viable el servicio en escenarios de alta concurrencia.

\subsection{Descripción del usuario}
El sistema centra su alcance en el estudiante de Ingeniería Civil en Informática y Telecomunicaciones de la Facultad de Ingeniería y Ciencias (FIC), cuyo avance curricular se rige por la malla 2020. La estructura curricular en que se desenvuelve el estudiante se encuentra caracterizado por ser híbrida y bajo un modelo de inscripción por ranking, donde el plan de estudio es organizado por tres fases que tienen distintas variables a la hora de planificación. Inicialmente, el estudiante transita por un ciclo de Plan Común, compuesto por asignaturas de ciencias básicas impartidas masivamente y compartidas con otras ingenierías, implicando mayor diversidad de posibles secciones a seleccionar. Posteriormente, se accede al ciclo de Especialidad, donde la oferta se reduce a asignaturas propias de la carrera, aumentando la frecuencia de topes horarios, menor cantidad de cursos disponibles y la dependencia de prerrequisitos.  Finalmente, la malla incorpora un ciclo transversal compuesto por los niveles de Inglés y los Cursos de Formación General (CFG), los cuales son impartidos a nivel universitario con otras facultades, por lo que contiene mayor diversidad de posibles cursos a tomar. Además de  poseer una libertad de inscripción casi total, pudiendo cursarse incluso hasta el último año debido a la carencia de prerrequisitos formales.\\

El proceso de toma de ramos opera mediante ventanas de tiempo segmentadas por ranking académico. Donde estudiantes con mayor ranking pueden acceder tempranamente a la oferta completa con distintas posibilidades de horario. Mientras que, un estudiante con ranking medio o bajo ingresa al sistema en un escenario de escasez, donde las secciones más demandadas por calidad docente o conveniencia horaria suelen estar agotadas. Comúnmente los estudiantes diseñan una planificación ideal que se puede volver inviable minutos antes de su turno de inscripción, forzando al estudiante a realizar una reingeniería improvisada de su horario, debiendo decidir cómo reorganizar su carga para evitar retrasos curriculares, muchas veces sacrificando sus preferencias personales o aceptando combinaciones ineficientes.\\

Frente a este escenario, el usuario debe gestionar variables que recaen más allá de la validación de conflictos de horario, priorizando:
\begin{itemize}
    \item \textbf{Gestión carga académica} Debido a que la malla permite adelantar o recuperar asignaturas, el estudiante busca equilibrar estratégicamente los cursos pendientes según sus prioridades.
    \item \textbf{Priorización de docente} El estudiante, puede buscar alternativas docentes en caso de que su primera opción no se encuentre disponible.
    \item \textbf{Preferencia horaria} Estudiante que deben compatibilizar sus estudios con responsabilidades laborales o personales, y requieren concentrar su actividad académica en bloques continuos personalizados.
\end{itemize}

Frente a esta dinámica se genera un escenario de alta incertidumbre, según reportes previos de usabilidad, el estudiante no solo enfrenta la complejidad combinatoria, sino también la ansiedad de operar contra reloj en un sistema que históricamente presentaba inestabilidad bajo carga. Por lo que, el usuario necesita de una herramienta de respuesta inmediata que le permita iterar escenarios 'Qué pasaría si...' en tiempo real durante su ventana de inscripción.
\subsection{Descripción del problema}
La problemática que motiva este proyecto surge de la necesidad de evolucionar las herramientas de planificación académica existentes. Se identifican dos brechas críticas que justifican la intervención: una funcional, relacionada con la desconexión entre el modelo de optimización clásico y las necesidades reales del estudiante, y una técnica, derivada de la obsolescencia de la arquitectura legada que será abordada con más detalle en las siguientes secciones.\\

La inscripción de asignaturas constituye un hito importante para los estudiantes de Ingeniería Civil en Informática y Telecomunicaciones, quienes deben navegar una malla curricular híbrida que combina cadenas de prerrequisitos rígidos con una oferta horaria fragmentada. En la actualidad, el proceso de decisión recae desproporcionadamente en el estudiante, quien debe procesar manual o semi-asistidamente una oferta académica en Excel, proceso en el que el estudiante debe cruzar su historial académico con la oferta vigente, verificar topes horarios mediante inspección visual y competir por cupos en tiempo real bajo un sistema de ranking, lo que eleva el riesgo de errores críticos como la omisión de asignaturas críticas o la generación de horarios incompatibles con su vida personal.\\

En cuanto a la herramienta de soporte legada, fue diseñada bajo la premisa de optimizar exclusivamente la ruta crítica, priorizando asignaturas que aceleran la titulación. Si bien este enfoque es matemáticamente correcto para minimizar la duración de la carrera, en la práctica resulta insuficiente para el perfil actual del estudiante. Al realizar su planificación, el alumno no solo busca la velocidad de egreso, sino que optimiza variables de "Calidad de Vida" tales como la gestión de carga académica (balancear ramos difíciles), la preferencia por docentes específicos y la compactación horaria para compatibilizar estudios con responsabilidades laborales. Al no incorporar estas variables multidimensionales, el estudiante se ve forzado a la autogestión frente a un espacio de búsqueda de 692 secciones, limitado a su capacidad de explorar todas las combinaciones posibles impidiendo obtener un horario ideal.\\

Por otro lado, la perspectiva técnica se encuentra ligado al sistema legado el cual alcanzó un límite de escalabilidad frente al volumen actual de la oferta académica. Enfrentando dos problemas principales, quiebre de integridad referencial producto de la modificación administrativa de los códigos de asignatura, que impide vincular el historial académico con la oferta vigente, y un agotamiento del modelo algorítmico, evidenciado por la incapacidad del motor actual para procesar la combinatoria de las secciones sin incurrir en timeouts. 

\subsection{Objetivos}
La presente sección define los objetivos generales y específicos que guían el desarrollo del presente proyecto, enfocándose en la optimización y reingeniería del sistema de Asignación de Horarios Académicos (ATR).
\subsubsection{Objetivo general}
Desarrollar y consolidar un motor de optimización de horarios que, preservando los aportes conceptuales de \texttt{RutaCritica}, entregue una solución de producción de alto rendimiento, baja latencia y reproducción verificable de resultados.

\subsubsection{Objetivos específicos}
Para alcanzar el objetivo general, se han definido los siguientes objetivos específicos, alineados con la continuidad conceptual desde \texttt{RutaCritica} hacia \texttt{Quickshift}:

\begin{enumerate}
    \item \textbf{Documentar y formalizar la herencia técnica:} Preservar y documentar los principios desarrollados en \texttt{RutaCritica} —análisis PERT, función de prioridad posicional (CC+UU+KK+SS) y modelado por grafos—, proporcionando pruebas de equivalencia y ejemplos reproducibles.

    \item \textbf{Reimplementar con garantías de producción:} Traducir los conceptos clave a una implementación en Rust (`Quickshift`) que conserve la semántica original donde aplica, y que introduzca mejoras de ingeniería para garantizar determinismo, robustez y testabilidad.

    \item \textbf{Mejorar rendimiento y escalabilidad:} Sustituir las búsquedas exactas no acotadas por una heurística constructiva voraz con diversificación por semillas (greedy multi-seed), apuntando a latencias en rango de tiempo real (objetivo: <200ms) y eliminación de timeouts bajo carga.

    \item \textbf{Normalización y equivalencias:} Implementar un sistema de normalización y mapeo de equivalencias entre códigos de asignatura para aumentar la cobertura (meta: >85\%) y evitar propuestas de cursos equivalentes ya aprobados por el estudiante.

    \item \textbf{Integridad curricular y filtros de usuario:} Mantener e integrar las comprobaciones PERT/prerrequisitos, junto con filtros y optimizaciones orientadas a la calidad de vida (compactación horaria, preferencia docente, minimización de gaps).

    \item \textbf{Evaluación reproducible y entrega:} Diseñar y publicar una batería de comparativas reproducibles frente a \texttt{RutaCritica} y la implementación legada (tiempos, memoria, tasa de soluciones válidas), junto con código, scripts y documentación para verificación externa.
\end{enumerate}